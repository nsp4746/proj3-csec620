\section{Introduction}

In today's digital world, phishing remains one of the most prevalent cybersecurity threats, targeting individuals,
organizations, and critical infrastructures. Phishing attacks employ deceptive tactics, such as fraudulent emails or
malicious links, to manipulate victims into disclosing sensitive information or performing unauthorized actions. Consequences
of these attacks can be devastating, regardless of victim. It ranges from financial loss to data breaches and reputational damage.
The increasing sophistication of phishing techniques necessitates the development of robust detection mechanisms.
Traditional approaches, such as rule-based filtering and signature matching, often struggle to keep pace with the
evolving nature of these attacks. As a result, researchers and practitioners have turned to machine learning (ML) and
deep learning (DL) techniques to enhance the detection capabilities against phishing attempts.
This research explores the development of a neural network-based model for detecting phishing data. This model also includes other
approaches to the dataset to evaluate the performance of different approaches.
Leveraging a dataset of 111 features extracted from emails and links, the model aims to distinguish phishing attempts
from legitimate communications with high accuracy. This paper provides an overview of phishing threats, discusses related
work in phishing detection using machine learning, and details the proposed neural network architecture,
its implementation,and performance evaluation. By advancing the application of deep learning in cybersecurity,this research seeks to contribute to the growing body of
work aimed at mitigating phishing threats and securing the digital ecosystem.